
\begin{table}
\begin{center}
\begin{tabular}{ccc} 
\toprule
Kernel    & Model                          & WAIC  \\ \midrule
Gaussian  & Base                           & 13115 \\
Gaussian  & Variable $\gamma$              & 13053 \\
Gaussian  & Variable $\psi$                & 13038 \\   
Gaussian  & Variable $\psi$ and $\gamma$   & 12953 \\
Power law & Base                           & 12853 \\
Power law & Variable $\gamma$              & 12789 \\
Power law & Variable $a$                   & 12702 \\
Power law & Variable $a$ and $b$           & $b$ EPC   \\
Power law & Variable $a$ and $\gamma$      & 12690 \\  
Power law & Variable $a$, $b$ and $\gamma$ & $b$ EPC   \\
\bottomrule
\end{tabular}
\caption{Wantanabe-Akaike Information Criterion (WAIC) values for
  eight candidate models. EPC denotes the exchangeable prior collapse
  for the indicated parameter. Models that use the Power law kernel
  have lower WAIC values.}
\end{center}
\label{table:WAIC}
\vspace{2cm}
\end{table}


% \begin{table}
% \begin{center}
% \begin{tabular}{ccccccccccccc} 
% \toprule
% Model & Ash & Beech & Birch & Elm & Hemlock & Maple & Oak & Other conifer &  Other hardwood & Pine & Spruce & Tamarack \\ \midrule
% Gaussian base      & multicolumn{12}{*}{0.21} \\   
% Power law base     & & multicolumn{12}{*}{0.07} \\   
% Gaussian variable  & 0.06 (0.01, 0.15) & 0.31 (0.13, 0.52) & 0.10 (0.05, 0.16) & 0.14 (0.07, 0.22) & 0.46 (0.37, 0.56) & 0.18 (0.10, 0.28) & 0.24 (0.18, 0.30) & 0.09 (0.01, 0.19) & 0.05 (0.01, 0.09) & 0.25 (0.21, 0.28) & 0.23 (0.13, 0.34) & 0.44 (0.23, 0.65) \\   
% Power law variable & 0.03 (3.55e-4, 0.11) & 0.17 (0.01, 0.49) & 0.03 (5.69e-4, 0.08) & 0.17 (0.09,0.24)& 0.27 (0.09, 0.45) & 0.20 (0.08, 0.30) & 0.05 (2.60e-3, 0.13) & 0.04 (3.23e-4, 0.15) & 0.05 (1.58e-3, 0.10) & 0.02 (3.11e-3, 0.05) & 0.14 (0.02, 0.29) & 0.37 (0.02, 0.62) \\   \bottomrule
% \end{tabular}
% \caption{Estimates of the local versus non-local parameter $\gamma$ for the four considered models, which include the gaussian kernel and power-law kernel base models, the variable gaussian model in which $\psi$ and $\gamma$ both vary by taxon, and the variable power law kernel model in which $a$ and $\gamma$ vary by taxon.}
% \end{center}
% \label{table:gamma}
% \vspace{2cm}
% \end{table}

\begin{table}
\begin{center}
\begin{tabular}{cccc} 
\toprule
               &                 & \multicolumn{2}{c}{Model} \\
Common name    & Scientific name & $\gamma$ (Variable Gaussian) & $\gamma$ (Variable Power law) \\  \midrule
           Ash & Fraxinus        & 6.24E-2 (7.15E-3, 1.64E-1)   & 3.87E-2 (3.55E-4, 1.07E-1)    \\
         Beech & Fagus           & 3.12E-1 (1.29E-1, 5.35E-1)   & 1.69E-1 (1.06E-2, 4.90E-1)    \\
         Birch & Betula          & 9.90E-2 (3.71E-2, 1.58E-1)   & 2.77E-2 (5.69E-4, 7.82E-2)    \\
           Elm & Ulmus           & 1.48E-1 (7.79E-2, 2.22E-1)   & 1.65E-1 (8.70E-2, 2.38E-1)    \\
       Hemlock & Tsuga           & 4.63E-1 (3.67E-1, 5.53E-1)   & 2.70E-1 (8.76E-2, 4.52E-1)    \\
         Maple & Acer            & 1.83E-1 (8.79E-2, 2.84E-1)   & 1.98E-1 (7.70E-2, 3.01E-1)    \\
           Oak & Quercus         & 2.41E-1 (1.82E-1, 3.03E-1)   & 5.16E-2 (2.60E-3, 1.26E-1)    \\
 Other Conifer & Abies           & 8.12E-2 (5.21E-3, 1.96E-1)   & 4.16E-2 (3.23E-4, 1.46E-1)    \\
Other Hardwood & $^{*}$ See caption    & 4.73E-2 (3.07E-3, 8.67E-2)   & 5.32E-2 (1.58E-3, 1.05E-1)    \\
          Pine & Pinus           & 2.49E-1 (2.14E-1, 2.82E-1)   & 2.11E-2 (3.12E-3, 4.86E-2)    \\
        Spruce & Picea           & 2.37E-1 (1.46E-1, 3.29E-1)   & 1.44E-1 (2.41E-2, 2.86E-1)    \\
      Tamarack & Larix           & 4.25E-1 (2.23E-1, 6.48E-1)   & 3.66E-1 (2.09E-2, 6.22E-1)    \\ \bottomrule
\end{tabular}
\caption{Estimates of parameter $\gamma$, indicating the proportion of locally sourced pollen for the variable Gaussian model in which $\psi$ and $\gamma$ both vary by taxon and the variable Power law kernel model in which $a$ and $\gamma$ vary by taxon. For the base models, $\gamma$ was estimate to be $0.21  (1.96E-1, 2.30E-1) $ for the Gaussian kernel model and $0.07  (3.79E-2, 9.53E-2) $ for the Power law kernel model. $^{*}$ The Other Hardwood grouping is composed of: Alnus, Carya, Cornus, Cupressaceae, Gleditsia, Liquidambar, Maclura, Morus, Nyssa, Ostrya carpinus, Platanus, Populus, Robinia, Rosaceae, Salix, Tilia, Zanthoxylum, as well as any undifferentiated hardwood (e.g. pollen grains classified as Fagus/Nyssa, Betula/Corylus, etc.).}
\end{center}
\label{table:gamma}
\vspace{2cm}
\end{table}


% \begin{table}
% \begin{center}
% \begin{tabular}{lcccc} 
% \toprule
%       & \multicolumn{4}{c}{Model} \\  \cmidrule(lr){2-5}
% Taxon & \multicolumn{2}{c}{Gaussian} & \multicolumn{2}{c}{Power law} \\  \cmidrule(lr){2-3}  \cmidrule(lr){4-5}
%       & base & variable & base & variable \\
% Ash            & 136 & 228 & 92 & 208 \\
% Beech          & 136 & 140 & 92 & 32 \\
% Birch          & 136 & 120 & 92 & 104 \\
% Elm            & 136 & 244 & 92 & 236 \\
% Hemlock        & 136 & 60  & 92 & 28 \\
% Maple          & 136 & 184 & 92 & 152 \\
% Oak            & 136 & 140 & 92 & 104 \\
% Other conifer  & 136 & 144 & 92 & 124 \\ 
% Other hardwood & 136 & 264 & 92 & 252 \\
% Pine           & 136 & 100 & 92 & 56 \\
% Spruce         & 136 & 168 & 92 & 132 \\
% Tamarack       & 136 &  96 & 92 & 120 \\ \bottomrule
% \end{tabular}
% \caption{}
% \end{center}
% \label{table:pollen_acc}
% \vspace{2cm}
% \end{table}



\begin{table}
\begin{center}
\begin{tabular}{lrrrr} 
\toprule
      & \multicolumn{4}{c}{Model} \\  \cmidrule(lr){2-5}
Taxon & \multicolumn{2}{c}{Gaussian} & \multicolumn{2}{c}{Power law} \\  \cmidrule(lr){2-3}  \cmidrule(lr){4-5}
      & 50\% & 90\% & 50\% & 90\% \\
\cmidrule(lr){2-3}  \cmidrule(lr){4-5}
Hemlock        & 60  & 280 & 28  & 292 \\
Tamarack       &  96 & 328 & 120 & 428 \\
Pine           & 100 & 216 & 56  & 332 \\
Birch          & 120 & 228 & 104 & 384 \\
Beech          & 140 & 340 & 32  & 300 \\
Oak            & 140 & 304 & 104 & 380 \\
Other conifer  & 144 & 276 & 124 & 396 \\ 
Spruce         & 168 & 356 & 132 & 408 \\
Maple          & 184 & 368 & 152 & 424 \\
Ash            & 228 & 404 & 208 & 452 \\
Elm            & 244 & 440 & 236 & 476 \\
Other hardwood & 264 & 448 & 252 & 476 \\
 \bottomrule
\end{tabular}
\caption{Radii (km) from a pollen source needed to capture 50\% and 90\% of
  the dispersed pollen for the variable gaussian and variable power
  law models. The gaussian base model estimated radii of 136 and 292
  km to capture 50\% and 90\% of the dispersed pollen across all taxa, while the power
  law model estimated radii of 92 and 376 km.}
\end{center}
\label{table:pollen_acc}
\vspace{2cm}
\end{table}
