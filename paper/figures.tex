
% %map showing core locations
% \begin{figure}
% \centering
% \includegraphics[width=7in]{figures/pollen_focal_scaled.pdf}
% \caption{}
% \label{fig:focal_scaled}
% \end{figure}

%Elicitation results
\begin{figure}
\centering
\begin{tabular}{c}
\includegraphics[width=6in]{figures/BROWNSBY_314.png} \\
\includegraphics[width=6in]{figures/PENNINGT_1884.png}
\end{tabular}
\caption{Pollen diagrams showing the relative pollen proportions as a
  function of core depth for the ten most common taxa for two sites in
  the calibration data set. Dashed horizontal lines indicate the
  samples selected by experts as pre-settlement samples. For some
  sites, such as Brownsby Lake, the settlement horizon is easily
  identified and all experts were in agreement with respect to its
  location (top panel). For other sites, such as Pennington Lake, the
  settlement horizon is difficult less obvious, resulting in multiple
  inferred pre-settlement samples (bottom panel).}
\label{fig:elicit}
\end{figure}


%PLS and pollen pie maps
\begin{figure}
\centering
\begin{tabular}{c}
\includegraphics[width=5in]{figures/pie_plot_pollen_ALL_v0_3.png} \\
\includegraphics[width=5in]{figures/pie_plot_pls_ALL_v0_3.png}
\end{tabular}
\caption{Pie maps depicting the relative composition of tree genera of pollen (top)
  and PLS vegetation (bottom) from the data. Note that the PLS data
  has been aggregated from 8 km to 34 km resolution for illustrative
  purposes.}
\label{fig:pie}
\end{figure}

% %trace plots
% \begin{figure}
% \centering
% %% \begin{tabular}{ccc}
% \includegraphics[width=5in]{figures/cal_trace.png}
% %\includepdf[nup=2x3]{figures/cal_trace.pdf}
%   %% \includegraphics[page=1,width=0.3\textwidth]{figures/cal_trace.png} &
%   %% \includegraphics[page=2,width=0.3\textwidth]{figures/cal_trace.png} &
%   %% \includegraphics[page=3,width=0.3\textwidth]{figures/cal_trace.png} \\
%   %% \includegraphics[page=4,width=0.3\textwidth]{figures/cal_trace.png} &
%   %% \includegraphics[page=5,width=0.3\textwidth]{figures/cal_trace.png} &
%   %% \includegraphics[page=6,width=0.3\textwidth]{figures/cal_trace.png} \\
%   %% \includegraphics[page=7,width=0.3\textwidth]{figures/cal_trace.png} &
%   %% \includegraphics[page=8,width=0.3\textwidth]{figures/cal_trace.png} &
%   %% \includegraphics[page=9,width=0.3\textwidth]{figures/cal_trace.png} \\
%   %% \includegraphics[page=10,width=0.3\textwidth]{figures/cal_trace.png} &
%   %% \includegraphics[page=11,width=0.3\textwidth]{figures/cal_trace.png} &
%   %% \includegraphics[page=12,width=0.3\textwidth]{figures/cal_trace.png} \\
%   %% \includegraphics[page=13,width=0.3\textwidth]{figures/cal_trace.png} &
%   %% \includegraphics[page=14,width=0.3\textwidth]{figures/cal_trace.png} &
%   %% \includegraphics[page=15,width=0.3\textwidth]{figures/cal_trace.png} 
% %% \end{tabular}
% \caption{Trace plots for the parameters in the gaussian dispersed
%   calibration model, including $\phi_k$ for $k=1, \ldots, W$, $\psi$,
%   and $\gamma$, as well as the joint log posterior,
%   $\textit{lp}$. Posterior estimates for each of the three runs are
%   depicted in shades of grey. The 2.5\% and 97.5\% quantiles are
%   indicated by the solid lines, while the 50\% quantiles are indicated
%   by the hashed lines.}
% \label{fig:trace}
% \end{figure}

% %veg and pollen heat maps: pine
% \begin{figure}
% \centering
% \includegraphics[width=7in]{figures/maps_compare_PINE_mod.png}
% \caption{Heat maps showing the relative abundances of pine in the PLS
%   composition data (top) and the pre-settlement era pollen data
%   (bottom).}
% \label{fig:compare_maps_PINE}
% \end{figure}

% %veg and pollen heat maps: birch
% \begin{figure}
% \centering
% \includegraphics[width=7in]{figures/maps_compare_BIRCH_mod.png}
% \caption{Heat maps showing the relative abundances of birch in the PLS
%   composition data (top) and the pre-settlement era pollen data
%   (bottom).}
% \label{fig:compare_maps_BIRCH}
% \end{figure}

%pollen raw versus scaled focal
\begin{figure}
\centering
\includegraphics[width=7in]{figures/pollen_phi_scaled_pl_Ka_Kgamma.png}
\caption{Pollen proportions plotted against local vegetation
  proportions (red crosses) or local vegetation proportion scaled by
  the pollen production factor $\bm{\phi}$ for the variable Power-law
  kernel model (black dots).}
\label{fig:focal_scaled}
\end{figure}

%pollen raw versus predicted
\begin{figure}
\centering
\includegraphics[width=7in]{figures/pollen_preds_pl_Ka_Kgamma.png}
\caption{Pollen proportions plotted against local vegetation
  proportions (red crosses) or model-predicted pollen for the variable
  Power-law kernel model (black dots).}
\label{fig:preds}
\end{figure}

%phi (differential production)
\begin{figure}
\centering
\includegraphics[width=7in]{figures/phi.png}
\caption{Mean values and 95\% credible intervals for differential
  production parameter $\phi$, by taxon for the four considered
  models.}
\label{fig:phi}
\end{figure}

%proportion pollen versus radius
\begin{figure}
\centering
\includegraphics[width=7in]{figures/kernel_discrete_cdfs.png}
\caption{The proportion of deposited (or accumulated) pollen plotted
  as a function of radius for each of the modelled taxa and for each
  of the four considered models (Gaussian base and variable; Power-law
  base and variable).}
\label{fig:cdf}
\end{figure}

%potential pollen maps by taxon
\begin{figure}
\centering
\includegraphics[width=7in]{figures/map_pls_pollen.png}
\caption{Heat maps of the PLS dat, pollen data, and predicted sediment
  pollen for the variable Gaussian (G) and Power-law (PL) models for a
  subset of modelled taxa.}
\label{fig:maps_pp}
\end{figure}

%PPE scatter
\begin{figure}
\centering
\includegraphics[width=4in]{figures/PPEs_panels.png}
\caption{Pollen productivity estimates (PPEs) from STEPPS versus sets
  of previously published PPEs from different geographical regions
  (Upper Midwest USA, Northeastern USA, European continent). Published
  PPE data sets were generated using linear regression (LR), extended
  R-value (ERV), or bias-corrected ERV models.}
\label{fig:ppe}
\end{figure}

%SV scatter
\begin{figure}
\centering
\includegraphics[width=7in]{figures/SVs_90_single.png}
\caption{Scatter plot of the STEPPS 90\% capture radius versus
  measured sedimentation velocities.}
\label{fig:svs}
\end{figure}

%STEPPS-predicted composition for the four models by taxon
\begin{figure}
\centering
\includegraphics[width=4.5in]{figures/maps_predicted_veg.png}
\caption{Heat maps of the STEPPS composition estimates, based on the
  Gaussian (G; base and variable) and Power-law (PL; base and
  variable) candidate calibration models.}
\label{fig:maps_stepps_veg}
\end{figure}

%PLS estimates, STEPPS estimates, and uncertainty for the best model
\begin{figure}
\centering
\includegraphics[width=3.5in]{figures/maps_PLS_STEPPS_SD.png}
\caption{Heat maps of the PLS composition estimates, STEPPS
  composition estimates based on the Variable Power-law (PL)
  calibration model and their uncertainty quantified as the standard
  deviation of the posterior estimates.}
\label{fig:maps_pls_stepps_sd}
\end{figure}

%Dissimilarity maps
\begin{figure}
\centering
\includegraphics[width=4in]{figures/maps_l2.png}
\caption{Heat maps of the dissimilarity as measured by the Euclidean
  distance between the PLS data and STEPPS composition estimates for
  different underlying calibration models.}
\label{fig:maps_l2}
\end{figure}



