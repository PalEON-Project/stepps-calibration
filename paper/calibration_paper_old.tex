\documentclass[12pt]{article}

\usepackage[margin=1in]{geometry}  % set the margins to 1in on all sides
\usepackage{graphicx}              % to include figures
\usepackage{amsmath, bm}               % great math stuff
\usepackage{amsfonts}              % for blackboard bold, etc
\usepackage{amsthm}                % better theorem environments

\newcommand{\kron}{\raisebox{1pt}{\ensuremath{\:\otimes\:}}} 
\newcommand{\lgamma}{\text{lgamma}} 
\newcommand{\digamma}{\text{digamma}} 

\begin{document}

\nocite{*}

\title{Prediction model for vegetation composition using pollen proxy data for the Upper Midwest}

\author{Andria Dawson}

\maketitle

\section{Introduction}

Understanding forest ecosystems of the past can provide us with valuable information about how ecosystems respond to biotic and abiotic factors. In particular, forest ecosystems respond to climatic variations that operate at different scales, and forest ecosystems of the past can inform us about the climate-forest relationship. 

Terrestrial ecosystem models describe the carbon, nitrogen and water dynamics of terrestrial plants and soils....   

\section{Data}

\subsection{Spatial domain}
Our study area is the Upper Midwestern US, which includes Minnesota, Wisconsin, and the upper peninsular of Michigan. The lower peninsula of Michigan was not included because: 1) it is spatially disjoint from the rest of the domain; and 2) the public land survey forest data is still in the process of being digitized, so the composition data available is incomplete. 

\subsection{Tree taxa}
We focus on a subset of taxa that are of particular interest, inlcuding the most abundant taxa and any taxa that are of ecological importance. Our modelled taxa includes: Ash, Beech, Birch, Elm, Hemlock, Maple, Oak, Pine, Spruce, Larch, as well as both Other Conifer and Other Hardwood which include those respective tree types not explicitly included elsewhere. Separating other hardwood and conifers as opposed to having a single other group was motivated by ecological modellers who are often interested in grouping taxa as deciduous, conifer, and deciduous conifers. Additionally, separating these groups allows the model to treat each group seperately, which may be benificial because of the inherent differences between conifer and deciduous seed production, although we recognize that the variability in production and dispersal within each of these groups is still large.

\subsection{Public Land Survey (PLS) data}
XXX

\subsection{Pollen data}
With the push for robust and reproducible research from the scientific community, paleoecoinformatics has responded with the development of tools that make accessing and using large datasets possible. One such tool that has made this work possible is the Neotoma database (\url{neotomadb.org}), which stores a variety of types of paleoecological data, including pollen data. Accessing this data can be done using the Neotoma API (), or using the R neotoma package \cite{neotoma}. Using these tools granted us access to 176 fossil pollen cores falling within our domain. Associated with each of these cores is a table containing counts by taxon for a series of depths. In this study, for each pollen core we are interested in the pre-settlement sample that is closest in time to the PLS data. Typically, age-depth models are used to assign ages to sample depths. However, there are many different types of age-depth models, each with its own set of benefits and shortcomings. Instead, we rely on a panel of experts to interpret patterns in the pollen count data to identify pre-settlement sample estimates.

\subsection{Expert elicitation of pre-settlement depth}
 As a result of European settlement, land clearances led to an increase in non-arboreal pollen. In the Upper Midwest, significant increases in Ambrosia, Rumex, and/or Poaceae are typically coincident with the settlement horizon. When these increases can be identified based on pollen count data, we can determine the pre-settlement sample - the sample that falls just before any increases in agricultural indicator species. In practice, identifying increases in agricultural indicators is often difficult, when possible, and can be subjective. We were interested in: 1) identifying the pre-settlement sample using consistent methodology, and 2) assessing the variability in assignment of pre-settlement among paleoecologists and palynologists. To address these questions, we asked a team of experts to identify pre-settlement samples. Experts were provided with pollen diagrams depicting proportional changes through time as a function of depth for key indicator species and the ten most abundant arboreal taxa. Experts were prohibited from relying on stratigraphic dates (radiocarbon or other) or age-depth model estimates of sample age. We also provided space so that experts could comment on their certainty of their pre-settlement sample. In the case that there was no distinguishable pre-settlement sample, experts were instructed to report NA.

\subsection{Calibration model}

Pollen counts are modelled using the dirichlet-multinomial to account overdispersion resulting from the sparsity and uneven distribution of sampled lakes throughout the domain. The precision parameter is the hadamard product of a simplex vector $\bm{r^{\textup{pond}}_i}$ that represents the relative proportions of tree taxa contributing to sediment pollen at lake $i$, and of a parameter vector $\bm{\phi}$ that accounts for differential pollen production between taxa. Pollen counts at pond $i$ are denoted by  $\bm{y}_i$, where 
\begin{align}
\bm{y}_i \sim DM (n_i, \bm{\phi} \cdot \bm{r^{\textup{pond}}_i}).
\label{eq:DM}
\end{align}

To determine which trees contribute to pollen sediment in lake $i$, we consider both local contributers from the cell $s(i)$ that contains lake $i$, and long-distance contributers from all other cells in the domain. The ratio of local to long-distance contributions is determined by the parameter $\gamma$. Then
\begin{align}
\bm{r^{\textup{pond}}_i} = \gamma r(s(i)) + (1-\gamma) C \sum_{s_k \neq s(i) } r(s_k) w(s(i), s_k)
\end{align}
where long-distance contributions are weighted by the isotropic weight function $w(s(i), s_k)$ which depends only on the distance between the pond cell $s(i)$ and the cell $s_k$. This weight function is defined using a gaussian dispersal kernal, written as
\begin{align}
w(s(i), s_k) = \exp\left( - \frac{d(s(i), s_k)^2}{\psi^2} \right).
\end{align}
where $d(s(i),s_k)$ is the distance between cells $s(i)$ and $s_k$, and $\psi$ is a parameter that describes the spread of the kernel. As expected, this kernel assigns less weight to those contributers that are farther away, although all weights are non-zero. Previous work has shown that pollen dispersal distances can reach up to XXXX, and that long-distance dispersal events may be rare, but have the ability to drive range expansion XXX. 

% Vegetation proportions are determined by the the underlying latent vegetation composition processes for each taxon, $\bm{g_p}$, using a sum-to-one constraint

% \begin{align*}
% r_p(s(\cdot)) = \frac{\exp(g_p(s(\cdot)))}{1 + \sum_{k=1}^W\exp(g_k(s(\cdot)) }
% \end{align*}
% For each taxon, the latent vegetation composition is modelled  Gaussian Process, where 

\section{Results}



\end{document}
