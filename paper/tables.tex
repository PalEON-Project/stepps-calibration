

% \pgfplotstableset{
% }


 % \begin{table}[h!]
 %  \begin{center}
 %    \caption{XXX.}
 %    \label{table1}



    \pgfplotstabletypeset[
      begin table=\begin{landscape}\begin{longtable},
      end table=\caption{A full list of sediment pollen sites in the %
                    Upper Midwestern USA considered for inclusion in %
                    the calibration data set. Meta data provided %
                    includes site name (Site), Neotoma or given %
                    identification number (ID), latitude (Lat) and %
                    longitude (Long), principal investigator (PI), %
                    depth of the settlement-era pollen sample (Depth; %
                    may be recorded as either depth from upper most %
                    sediment layer, or from lake surface), an %
                    indication if the dataset came from Neotoma %
                    (Neotoma), an indication as to whether the dataset %
                    is included in the final calibration data set %
                    (Calibration), and in the case a site is not %
                    included (Calibration value of N), then a note %
                    indication the reason for inadmissibility %
                    (Notes). The reasons for inadmissibility are: A) %
                    Three or four experts did not assign a %
                    pre-settlement sample; B) Two experts did not %
                    assign a pre-settlement sample, and the two %
                    assigned pre-settlement samples were at least 300 %
                    years away from 1850 according to the default %
                    age-depth models; C) No core top.} %
                    \label{table:site-data}\end{longtable}\end{landscape},
      multicolumn names,
      col sep=comma,
      font=\small,
      display columns/0/.style={string type},
      display columns/1/.style={string type},
      display columns/4/.style={string type},
      display columns/6/.style={string type},
      display columns/7/.style={string type},
      display columns/8/.style={string type},
      empty cells with=\ ,
      every head row/.style={
		before row={\toprule},
		after row={\midrule\endhead}},
      every last row/.style={after row={\bottomrule\\}},
    ]{../../stepps-data/data/meta_v3.csv}

\begin{table}
\begin{center}
\begin{tabular}{cccc} 
\toprule
Kernel    & Model                          & WAIC           & Dissimilarity \\ \midrule
Gaussian  & Base                           & 13115          & 2062 \\
Gaussian  & Variable $\gamma$              & 13053          &  -   \\
Gaussian  & Variable $\psi$                & 13038          &  -   \\   
Gaussian  & Variable $\psi$ and $\gamma$   & 12953          & 2078 \\
Power-law & Base                           & 12852          & \textbf{1979} \\
Power-law & Variable $\gamma$              & 12789          &  -   \\
Power-law & Variable $a$                   & 12702          &  -   \\
Power-law & Variable $a$ and $b$           & $b$ EPC        &  -   \\
Power-law & Variable $a$ and $\gamma$      & \textbf{12689} & 2044 \\  
Power-law & Variable $a$, $b$ and $\gamma$ & $b$ EPC        &  -   \\
\bottomrule
\end{tabular}
\caption{Goodness-of-fit measures for the calibration and prediction
  model for different underlying dispersal submodels. For the
  calibration model, we report the Wantanabe-Akaike Information
  Criterion (WAIC). EPC denotes an exchangeable prior collapse for the
  indicated parameter. Overall, calibration models with the Power-law
  kernel have lower WAIC values. For the prediction model, we report
  the dissimilarity between the PLS data and the predictions as
  measured by the sum of the cell-wise Euclidean distances.}
\label{table:GOF}
\end{center}
\vspace{2cm}
\end{table}

\begin{landscape}
\begin{table}
\begin{center}
\begin{tabular}{ll*{6}{r@{ (}r@{, }r}}
\toprule
                             &  & \multicolumn{18}{c}{Model} \\ \cmidrule(lr){3-20}
\multicolumn{2}{c}{Taxon names} & \multicolumn{9}{c}{Gaussian} & \multicolumn{9}{c}{Power law} \\  \cmidrule(lr){1-2}  \cmidrule(lr){3-11}  \cmidrule(lr){12-20}
Common & Scientific             & \multicolumn{3}{c}{50\%} & \multicolumn{3}{c}{70\%} & \multicolumn{3}{c}{90\%} & \multicolumn{3}{c}{50\%} & \multicolumn{3}{c}{70\%} & \multicolumn{3}{c}{90\%} \\
\cmidrule(lr){1-1} \cmidrule(lr){2-2} \cmidrule(lr){3-5} \cmidrule(lr){6-8} \cmidrule(lr){9-11} \cmidrule(lr){12-14} \cmidrule(lr){15-17} \cmidrule(lr){18-20}
% Hemlock        & 60  & 280 & 28  & 292 \\
% Tamarack       &  96 & 328 & 120 & 428 \\
% Pine           & 100 & 216 & 56  & 332 \\
% Birch          & 120 & 228 & 104 & 384 \\
% Beech          & 140 & 340 & 32  & 300 \\
% Oak            & 140 & 304 & 104 & 380 \\
% Other conifer  & 144 & 276 & 124 & 396 \\ 
% Spruce         & 168 & 356 & 132 & 408 \\
% Maple          & 184 & 368 & 152 & 424 \\
% Ash            & 228 & 404 & 208 & 452 \\
% Elm            & 244 & 440 & 236 & 476 \\
% Other hardwood & 264 & 448 & 252 & 476 \\
% Hemlock& Tsuga & 60 &   8 & 104) & 284 & 248 & 332)&32 &  16 &  44) & 372 & 335 & 412) \\
% Tamarack& Larix & 72 &   8 & 192) & 288 & 162 & 574)&40 &   8 & 192) & 442 & 281 & 582) \\
% Pine& Pinus & 96 &  88 & 108) & 216 & 198 & 236)&76 &  60 &  88) & 444 & 408 & 464) \\
% Birch& Betula & 120 & 108 & 136) & 232 & 208 & 264)&120 &  92 & 152) & 496 & 460 & 525) \\
% Beech& Fagus & 140 &   8 & 184) & 348 & 276 & 428)&40 &   8 &  69) & 386 & 324 & 441) \\
% Oak& Quercus & 140 & 124 & 154) & 304 & 280 & 340)&116 &  99 & 140) & 492 & 470 & 512) \\
% Fir& Abies & 144 & 112 & 188) & 276 & 218 & 354)&140 &  71 & 209) & 512 & 438 & 560) \\
% Spruce& Picea & 168 & 136 & 208) & 368 & 312 & 440)&144 & 100 & 197) & 524 & 484 & 564) \\
% Maple& Acer & 176 & 108 & 296) & 368 & 232 & 566)&148 &  52 & 296) & 536 & 415 & 612) \\
% Ash& Fraxinus & 232 & 172 & 296) & 436 & 330 & 540)&236 & 167 & 337) & 576 & 535 & 617) \\
% Elm& Ulmus & 256 & 196 & 324) & 500 & 396 & 592)&272 & 196 & 341) & 600 & 562 & 632) \\
% Other hardwood& * & 284 & 244 & 332) & 520 & 460 & 574)&312 & 252 & 352) & 608 & 584 & 624) \\
Hemlock& Tsuga & 60 &   8 & 104) & 168 & 136 & 198) & 284 & 248 & 332)&32 &  16 &  44) & 110 &  84 & 140) & 372 & 335 & 412) \\
Tamarack & Larix & 72 &   8 & 192) & 172 &  74 & 374) & 288 & 162 & 574)&40 &   8 & 192) & 160 &  47 & 369) & 442 & 281 & 582) \\
Pine & Pinus & 96 &  88 & 108) & 148 & 132 & 160) & 216 & 198 & 236)&76 &  60 &  88) & 184 & 156 & 204) & 444 & 408 & 464) \\
Birch & Betula & 120 & 108 & 136) & 164 & 148 & 184) & 232 & 208 & 264)&120 &  92 & 152) & 248 & 207 & 289) & 496 & 460 & 525) \\
Beech & Fagus & 140 &   8 & 184) & 228 & 178 & 276) & 348 & 276 & 428)&40 &   8 &  69) & 120 &  75 & 177) & 386 & 324 & 441) \\
Oak & Quercus & 140 & 124 & 154) & 208 & 188 & 228) & 304 & 280 & 340)&116 &  99 & 140) & 244 & 215 & 273) & 492 & 470 & 512) \\
Fir & Abies & 144 & 112 & 188) & 196 & 156 & 252) & 276 & 218 & 354)&140 &  71 & 209) & 272 & 175 & 349) & 512 & 438 & 560) \\
Spruce & Picea & 168 & 136 & 208) & 252 & 210 & 304) & 368 & 312 & 440)&144 & 100 & 197) & 284 & 231 & 352) & 524 & 484 & 564) \\
Maple & Acer & 176 & 108 & 296) & 256 & 158 & 410) & 368 & 232 & 566)&148 &  52 & 296) & 304 & 151 & 449) & 536 & 415 & 612) \\
Ash& Fraxinus & 232 & 172 & 296) & 312 & 234 & 396) & 436 & 330 & 540)&236 & 167 & 337) & 380 & 304 & 469) & 576 & 535 & 617) \\
Elm& Ulmus & 256 & 196 & 324) & 356 & 276 & 444) & 500 & 396 & 592)&272 & 196 & 341) & 424 & 346 & 492) & 600 & 562 & 632) \\
Other hardwood& * &284 & 244 & 332) & 380 & 330 & 434) & 520 & 460 & 574)&312 & 252 & 352) & 448 & 399 & 484) & 608 & 584 & 624) \\
 \bottomrule
\end{tabular}
\caption{Radii (km) from a pollen source needed to capture 50, 70 and
  90\% of the dispersed pollen for the variable gaussian and variable
  power law models. The gaussian base model estimated radii of $136\,
  (132, 144)$, $200\, (192, 212)$ and $292\, (276, 308)$ km to capture
  50, 70 and 90\% of the dispersed pollen across all taxa, while the
  power law model estimated radii of $108\, (100, 116)$, $234\, (220,
  244)$ and $488\, (476, 496)$ km.  $^{*}$ The Other Hardwood grouping
  is composed of: Alnus, Carya, Cornus, Cupressaceae, Gleditsia,
  Liquidambar, Maclura, Morus, Nyssa, Ostrya carpinus, Platanus,
  Populus, Robinia, Rosaceae, Salix, Tilia, Zanthoxylum, as well as
  any undifferentiated hardwood (e.g. pollen grains classified as
  Fagus\textbackslash Nyssa, Betula\textbackslash Corylus, etc.).}
\end{center}
\label{table:pollen_acc}
\vspace{2cm}
\end{table}
\end{landscape}